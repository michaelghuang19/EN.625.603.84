\documentclass{article}
\linespread{1.3}
\usepackage[margin=50pt]{geometry}
\usepackage{amsmath, amsthm, amssymb, amsthm, tikz, fancyhdr, graphicx, systeme}
\pagestyle{fancy}
\renewcommand{\headrulewidth}{0pt}
\newcommand{\changefont}{\fontsize{15}{15}\selectfont}

\fancypagestyle{firstpageheader}{
  \fancyhead[R]{
    \changefont
    \parbox[t]{4cm}{ % Adjust width as needed
      Michael Huang\\
      EN.625.603.84\\
      Problem Set \#8
    }
  }
}

\begin{document}

\thispagestyle{firstpageheader}
{\Large 

\section*{11.2.1}
Crickets make their chirping sound by sliding one wing cover very rapidly back and forth over the other. Biologists have long been aware that there is a linear relationship between \(temperature\) and the \(frequency\) with which a cricket chirps, although the slope and y-intercept of the relationship vary from species to species. The following table lists fifteen frequency-temperature observations recorded for the striped ground cricket, \(Nemobius ~ fasciatus ~ fasciatus\) (145). Plot these data and find the equation of the least squares line, y = a+bx. Suppose a cricket of this species is observed to chirp eighteen times per second. What would be the estimated temperature?
\\
\\


\section*{11.2.2}
The aging of whisky in charred oak barrels brings about a number of chemical changes that enhance its taste and darken its color. The following table shows the change in a whisky's proof as a function of the number of years it is stored (168).
\\
\\


\section*{11.2.3}
As water temperature increases, sodium nitrate (NaN\(\text{O}_3\)) becomes more soluble. The following table (110) gives the number of parts of sodium nitrate that dissolve in one hundred parts of water. Calculate the residuals, \(y_1 - \hat{y}_1, \dots, y_9 - \hat{y}_9\), and draw the residual plot. Does it suggest that fitting a straight line through these data would be appropriate?
\\
\\


\section*{11.2.7}
The relationship between school funding and student performance continues to be a hotly debated political and philosophical issue. Typical of the data available are the following figures, showing the per-pupil expenditures and graduation rate for twenty-six randomly chosen districts in Massachusetts. Graph the data and superimpose the least squares line, \(y = a + bx\).


\section*{11.2.13}
Prove that a least squares straight line must necessarily
pass through the point \(( \bar{x}, \bar{y})\).
\\
\\


\section*{11.2.14}
In some regression situations, there are \(a ~ priori\) reasons for assuming that the \(xy\)-relationship being approximated passes through the origin. If so, the equation to be fit to the \((x_i, y_i)\)'s has the form \(y = bx\). Use the least squares criterion to show that the “best” slope in that case is given by 
\[
b = \frac{\sum_{i=1}^n x_i y_i}{\sum_{i=1}^{n} x_i^2}
\]
\\
\\


\section*{11.3.2}
The best straight line through the Massachusetts funding/graduation rate data described in Question 11.2.7
has the equation y = 81.088+0.412x, where s = 11.78848.

\subsection*{(a)} 
Construct a 95\% confidence interval for \(\beta_1\).
\\
\\


\subsection*{(b)}
What does your answer to part (a) imply about the outcome of testing \(H_0: \beta_1 = 0\) versus \(H_1: \beta_1 \neq 0\) at the \(\alpha = 0.05\) level of significance?
\\
\\


\section*{11.3.14}
Construct a 90\% confidence interval for \(\sigma^2\) in the cigarette-consumption/CHD mortality data given in Case Study 11.3.1.
\\
\\


\section*{11.3.16}
Regression techniques can be very useful in situationswhere one variable—say, \(y\)—is difficult to measure but \(x\) is not. Once such an \(xy\)-relationship has been “calibrated," based on a set of \((x_i, y_i)\)'s, future values of \(Y\) can be easily estimated using \(\hat{\beta}_0 + \hat{\beta}_1x\). Determining the volume of an irregularly shaped object, for example, is often difficult, but weighing that object is likely to be easy. The following table shows the weights (in kilograms) and the volumes (in cubic decimeters) of eighteen children between the ages of five and eight (15). The estimated regression line has the equation \(y = -0.104+0.988x\), where \(s = 0.202\).

\subsection*{(a)} 
Construct a 95\% confidence interval for \(E(Y | 14.0)\).
\\
\\

\subsection*{(b)} 
Construct a 95\% prediction interval for the volume of a child weighing 14.0 kilograms.
\\
\\


\section*{11.3.17}
Construct a 95\% confidence interval for \(E(Y | 2.750) \) using the connecting rod data given in Case Study 11.2.1.
\\
\\


\section*{11.4.12}
Many people believe that a salary bonus is a reward for good performance. The corporate world may have a different understanding. A random sample of thirty chief executive officers of large capitalization public companies recorded the cash bonus paid, \(x\) (in \$100,000), and the performance of the company, \(y\), as measured by percentage change in company revenues. The following sums resulted. Find the sample coefficient of correlation. What does this study say about the relationship between bonuses and performance?
\\
\\

\section*{11.4.15}
A common saying in golf is “You drive for show,but you putt for dough.” To see if there is any truth in this assertion, data for ninety-six top money-winning golfers were examined. For each, their money earnings in 2014 (\(y\), in \$ millions), their average yards per drive (\(v\)), and their average number of putts (\(x\)) were tallied.

\subsection*{(a)} 
Show that the correlation coefficient between the
putting average and earnings reveal a slightly stronger relationship than that for driving and earnings.
\\
\\


\subsection*{(b)} 
For each correlation \(r\), compute \(r^2\) to show that neither the \(v\) nor the \(x\) variable alone is a good predictor of earnings.
\\
\\


% End of the large subsection
}

\end{document}